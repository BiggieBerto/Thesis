%%=============================================================================
%% Methodologie
%%=============================================================================

\chapter{\IfLanguageName{dutch}{Methodologie}{Methodology}}%
\label{ch:methodologie}

%% TODO: In dit hoofstuk geef je een korte toelichting over hoe je te werk bent
%% gegaan. Verdeel je onderzoek in grote fasen, en licht in elke fase toe wat
%% de doelstelling was, welke deliverables daar uit gekomen zijn, en welke
%% onderzoeksmethoden je daarbij toegepast hebt. Verantwoord waarom je
%% op deze manier te werk gegaan bent.
%% 
%% Voorbeelden van zulke fasen zijn: literatuurstudie, opstellen van een
%% requirements-analyse, opstellen long-list (bij vergelijkende studie),
%% selectie van geschikte tools (bij vergelijkende studie, "short-list"),
%% opzetten testopstelling/PoC, uitvoeren testen en verzamelen
%% van resultaten, analyse van resultaten, ...
%%
%% !!!!! LET OP !!!!!
%%
%% Het is uitdrukkelijk NIET de bedoeling dat je het grootste deel van de corpus
%% van je bachelorproef in dit hoofstuk verwerkt! Dit hoofdstuk is eerder een
%% kort overzicht van je plan van aanpak.
%%
%% Maak voor elke fase (behalve het literatuuronderzoek) een NIEUW HOOFDSTUK aan
%% en geef het een gepaste titel.

\section{Phase 1: Literary review}
The research will be divided into 4 phases, the first of which is a literature review. This will be performed to examine existing approaches and gain deeper insights into the evolution of CI/CD practices.

\section{Phase 2: Design}
Following the literature review, the next step involves designing a comprehensive framework for evaluating the sustainability of CI/CD practices. This framework will be used to assess the environmental impact, efficiency, and scalability of various CI/CD practices. The design phase will also involve the identification of key sustainability metrics and performance indicators that will be used to evaluate the effectiveness of each approach.

\subsection{Phase 3: Proof-of-Concept}
The subsequent phase of the bachelor thesis involves crafting a comprehensive plan and a Proof-of-Concept that delineates the objectives, scope, and metrics to be assessed. Criteria for success will be defined to evaluate the effectiveness of each shortlisted approach. Subsequently, the Proof-of-Concept will be implemented within a virtual enviornment. This implementation phase will enable the measurement and collection of data pertaining to environmental impact, efficiency, and other relevant metrics. Concurrently, this initiative will aid Wolters Kluwer in advancing towards more sustainable and efficient CI/CD practices.

\subsection{Phase 4: Conclusion}
The findings from the comparative study of each shortlisted approach will be summarized to draw conclusions regarding the most effective and practical, sustainable CI/CD practices. This summarization will enable the provision of recommendations for organizations, such as my internship company, that are seeking to implement sustainable CI/CD processes. These recommendations will be based on the identified strengths and weaknesses of each approach, facilitating informed decision-making and effective implementation strategies for sustainable CI/CD practices.

