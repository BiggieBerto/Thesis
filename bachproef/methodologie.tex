%%=============================================================================
%% Methodologie
%%=============================================================================

\chapter{\IfLanguageName{dutch}{Methodologie}{Methodology}}%
\label{ch:methodologie}

%% TODO: In dit hoofstuk geef je een korte toelichting over hoe je te werk bent
%% gegaan. Verdeel je onderzoek in grote fasen, en licht in elke fase toe wat
%% de doelstelling was, welke deliverables daar uit gekomen zijn, en welke
%% onderzoeksmethoden je daarbij toegepast hebt. Verantwoord waarom je
%% op deze manier te werk gegaan bent.
%% 
%% Voorbeelden van zulke fasen zijn: literatuurstudie, opstellen van een
%% requirements-analyse, opstellen long-list (bij vergelijkende studie),
%% selectie van geschikte tools (bij vergelijkende studie, "short-list"),
%% opzetten testopstelling/PoC, uitvoeren testen en verzamelen
%% van resultaten, analyse van resultaten, ...
%%
%% !!!!! LET OP !!!!!
%%
%% Het is uitdrukkelijk NIET de bedoeling dat je het grootste deel van de corpus
%% van je bachelorproef in dit hoofstuk verwerkt! Dit hoofdstuk is eerder een
%% kort overzicht van je plan van aanpak.
%%
%% Maak voor elke fase (behalve het literatuuronderzoek) een NIEUW HOOFDSTUK aan
%% en geef het een gepaste titel.

\section{Phase 1: Literary review}
The research will be divided into 4 phases, the first of which is a literature review. This will be performed to examine existing approaches and gain deeper insights into the evolution of CI/CD practices.

\section{Phase 2: Metrics and Tools Analysis}
This phase will delve into the design of the research, focusing on the methodology and the approach taken to address the research questions. The chapter will also provide an overview of the research design, the research methods, and the data collection techniques used to address the research questions.

\section{Phase 3: Case Study}
The subsequent phase of the bachelor thesis involves a case study at Wolters Kluwer to evaluate the environmental impact of the current CI/CD practices. This case study will provide insight into the environmental impact by analyzing the energy consumption, carbon footprint, and resource utilization of the CI/CD practices at Wolters Kluwer. Additionally, the case study will evaluate the efficiency and scalability of these CI/CD practices.

\section{Phase 4: Comparative Study}
The fourth phase of the research will involve a comparative study between Windows and Linux agents. This phase will compare the energy-efficiency and performance of Windows and Linux agents in a TeamCity instance. 

\section{Phase 5: Conclusion}
The findings from the comparative study of each shortlisted approach will be summarized to draw conclusions regarding the most effective and practical, sustainable CI/CD practices. This summarization will enable the provision of recommendations for organizations, such as my internship company, that are seeking to implement sustainable CI/CD processes. These recommendations will be based on the identified strengths and weaknesses of each approach, facilitating informed decision-making and effective implementation strategies for sustainable CI/CD practices.

