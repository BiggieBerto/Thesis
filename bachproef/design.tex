%%=============================================================================
%% Design
%%=============================================================================

\chapter{Design}%
\label{ch:design}

\section{Introduction}%
\label{sec:introduction-design}

This chapter will delve into the design of the research, focusing on the methodology and the approach taken to address the research questions. The chapter will also provide an overview of the research design, the research methods, and the data collection techniques used to address the research questions. For the purpose of this research, we will be using a comparative analysis to evaluate the environmental impact of CI/CD practices. This analysis will be conducted by two methods: a case study at Wolters Kluwer and a Proof-of-Concept (PoC) implementation.

\section{Case Study at Wolters Kluwer}%
\label{sec:case-study}
The case study will give insight into the environmental impact of the current CI/CD practices at Wolters Kluwer. The case study will be conducted by analyzing the energy consumption, carbon footprint, and resource utilization of the CI/CD practices at Wolters Kluwer. The case study will also evaluate the efficiency and scalability of the CI/CD practices at Wolters Kluwer.

\section{Proof-of-Concept (PoC)}%
\label{sec:poc}
The PoC implementation will involve the implementation of the CI/CD best practices listed in the literature review. The PoC will be implemented in a virtual environment to measure the environmental impact, efficiency, and scalability of the CI/CD practices. 

\section{Metrics}%
\label{sec:metrics}

Before we can test the environmental impact of CI/CD practices, we need to define the metrics that will be used to measure this impact. The metrics will be used to evaluate the environmental impact of CI/CD practices and to compare the sustainability of different CI/CD practices. The metrics will be used to measure the energy consumption, carbon footprint, and resource utilization of CI/CD practices. The metrics will also be used to evaluate the efficiency and scalability of CI/CD practices.

The tool for the measurement of the metrics is Prometheus. We will use Prometheus to collect the data and Grafana to visualize the data. The data will be collected from the CI/CD pipeline and the PoC implementation. 

\section{Best Practices}%
\label{sec:best-practices}

The literature review has identified several best practices for sustainable CI/CD practices. These best practices will be implemented in the PoC to evaluate their environmental impact, efficiency, and scalability. The best practices include the use of energy-efficient hardware, the optimization of CI/CD pipelines, the use of renewable energy sources, and the reduction of resource utilization.

The following best practices will be implemented in the PoC:
\begin{itemize}
    \item Code optimization: The code will be optimized to reduce the energy consumption and resource utilization of the CI/CD pipeline.
    \item Clean-up CI/CD environment: The CI/CD environment will be cleaned up to reduce the energy consumption and resource utilization of the CI/CD pipeline.
    \item Containerization: The CI/CD pipeline will be containerized to reduce the energy consumption and resource utilization of the CI/CD pipeline.
    \item Linux vs Windows: The CI/CD pipeline will be tested on Linux and Windows to evaluate the environmental impact of the operating system.
\end{itemize}



