\documentclass[english,dit,thesis]{hogentreport}

\usepackage{lipsum} % For blind text, can be removed after adding actual content

%% Pictures to include in the text can be put in the graphics/ folder
\graphicspath{{graphics/}}

%% For source code highlighting, requires pygments to be installed
%% Compile with the -shell-escape flag!
\usepackage[section]{minted}

\usemintedstyle{solarized-light}
\definecolor{bg}{RGB}{253,246,227} %% Set the background color of the codeframe

%% Change this line to edit the line numbering style:
\renewcommand{\theFancyVerbLine}{\ttfamily\scriptsize\arabic{FancyVerbLine}}

%% Macro definition to load external java source files with \javacode{filename}:
\newmintedfile[javacode]{java}{
    bgcolor=bg,
    fontfamily=tt,
    linenos=true,
    numberblanklines=true,
    numbersep=5pt,
    gobble=0,
    framesep=2mm,
    funcnamehighlighting=true,
    tabsize=4,
    obeytabs=false,
    breaklines=true,
    mathescape=false
    samepage=false,
    showspaces=false,
    showtabs =false,
    texcl=false,
}

% Other packages not already included can be imported here

%%---------- Document metadata -------------------------------------------------
% TODO: Replace this with your own information
\author{Robbe Cherlet}
\supervisor{Mevr. C. Teerlinck}
\cosupervisor{Dhr. J. Delamper}
\title[A Comparative Analysis and Proof of Concept]%
    {Sustainability in Continuous Integration en Deployment-processes}
\academicyear{\advance\year by -1 \the\year--\advance\year by 1 \the\year}
\examperiod{1}
\degreesought{\IfLanguageName{dutch}{Professionele bachelor in de toegepaste informatica}{Bachelor of applied computer science}}
\partialthesis{false} %% To display 'in partial fulfilment'
%\institution{Internshipcompany BVBA.}

%% Add global exceptions to the hyphenation here
\hyphenation{back-slash}

%% The bibliography (style and settings are  found in hogentthesis.cls)
\addbibresource{bachproef.bib}            %% Bibliography file
\addbibresource{../voorstel/voorstel.bib} %% Bibliography research proposal
\defbibheading{bibempty}{}

%% Prevent empty pages for right-handed chapter starts in twoside mode
\renewcommand{\cleardoublepage}{\clearpage}

\renewcommand{\arraystretch}{1.2}

%% Content starts here.
\begin{document}

%---------- Front matter -------------------------------------------------------

\frontmatter

\hypersetup{pageanchor=false} %% Disable page numbering references
%% Render a Dutch outer title page if the main language is English
\IfLanguageName{english}{%
    %% If necessary, information can be changed here
    \degreesought{Professionele Bachelor toegepaste informatica}%
    \begin{otherlanguage}{dutch}%
       \maketitle%
    \end{otherlanguage}%
}{}

%% Generates title page content
\maketitle
\hypersetup{pageanchor=true}

%%=============================================================================
%% Voorwoord
%%=============================================================================

\chapter*{\IfLanguageName{dutch}{Woord vooraf}{Preface}}%
\label{ch:voorwoord}

%% TODO:
%% Het voorwoord is het enige deel van de bachelorproef waar je vanuit je
%% eigen standpunt (``ik-vorm'') mag schrijven. Je kan hier bv. motiveren
%% waarom jij het onderwerp wil bespreken.
%% Vergeet ook niet te bedanken wie je geholpen/gesteund/... heeft

I can proudily say that I have successfully completed my bachelor's thesis. This thesis is the result of hard work, dedication, and perseverance. I am happy to have had the opportunity to conduct this research on two subjects that I am passionate about: Continuous Integration and Deployment (CI/CD) and sustainability.
I would like to express my gratitude to my supervisor, Chantal Teerlinck, for her guidance, support, and valuable feedback throughout the research process. She pushed me through a period of busyness and I will not forget her saying "You have to pass your thesis" all the time. I would also like to thank my internship company, Wolters Kluwer, and mainly my co-supervisor, Jan Delamper, for providing me with the opportunity to conduct this research and for their support and encouragement. 
I would like to extend my heartfelt thanks to my friends and peers for their unwavering support and encouragement. Last but not least, I would like to thank my father with improving my languague and to proofread my thesis. 
I hope that this research will contribute to the field of CI/CD and sustainability and inspire others to explore this topic further.
%%=============================================================================
%% Samenvatting
%%=============================================================================

% TODO: De "abstract" of samenvatting is een kernachtige (~ 1 blz. voor een
% thesis) synthese van het document.
%
% Een goede abstract biedt een kernachtig antwoord op volgende vragen:
%
% 1. Waarover gaat de bachelorproef?
% 2. Waarom heb je er over geschreven?
% 3. Hoe heb je het onderzoek uitgevoerd?
% 4. Wat waren de resultaten? Wat blijkt uit je onderzoek?
% 5. Wat betekenen je resultaten? Wat is de relevantie voor het werkveld?
%
% Daarom bestaat een abstract uit volgende componenten:
%
% - inleiding + kaderen thema
% - probleemstelling
% - (centrale) onderzoeksvraag
% - onderzoeksdoelstelling
% - methodologie
% - resultaten (beperk tot de belangrijkste, relevant voor de onderzoeksvraag)
% - conclusies, aanbevelingen, beperkingen
%
% LET OP! Een samenvatting is GEEN voorwoord!

%%---------- Nederlandse samenvatting -----------------------------------------
%
% TODO: Als je je bachelorproef in het Engels schrijft, moet je eerst een
% Nederlandse samenvatting invoegen. Haal daarvoor onderstaande code uit
% commentaar.
% Wie zijn bachelorproef in het Nederlands schrijft, kan dit negeren, de inhoud
% wordt niet in het document ingevoegd.

\IfLanguageName{english}{%
\selectlanguage{dutch}
\chapter*{Samenvatting}

\section{Inleiding}
Deze bachelorproef richt zich op de huidige CI/CD-praktijken (Continuous Integration/Continuous Delivery) bij Wolters Kluwer. Het doel is om de milieu-impact van deze praktijken te evalueren en optimalisaties voor te stellen.

\section{Onderzoeksvraag}
De centrale onderzoeksvraag luidt: "Hoe kunnen de CI/CD-praktijken worden geoptimaliseerd om hun milieu-impact te minimaliseren?" De doelstelling is om inzicht te krijgen in het energieverbruik en grondstofgebruik van de CI/CD-praktijken, en aanbevelingen te doen voor verbetering.

\section{Methodologie}
Het onderzoek werd uitgevoerd door een gedetailleerde casestudy bij Wolters Kluwer. Er werd data verzameld over CPU-gebruik, geheugenverbruik, bouwstatistieken, grondstofgebruik en systeem-uptime van de CI/CD-pipeline. Vervolgens werden de beste praktijken geïmplementeerd en geëvalueerd. Ook werd er een vergelijking gemaakt tussen Linux en Windows agents.

\section{Resultaten}
De belangrijkste resultaten tonen aan dat er heel wat mogelijkheden zijn om de CI/CD-praktijken te optimaliseren. 

\section{Conclusie}
Het onderzoek toont aan dat het optimaliseren van CI/CD-praktijken niet alleen de efficiëntie verhoogt, maar ook de milieu-impact vermindert. Het implementeren van de voorgestelde beste praktijken kan leiden tot duurzamere softwareontwikkeling bij Wolters Kluwer. Het testen van best practices is niet gemakkelijk op een omgeving die niet volledig is geïsoleerd van andere processen. Het is belangrijk om de resultaten van dit onderzoek te zien als een eerste stap in de richting van een duurzamere softwareontwikkeling.



\selectlanguage{english}
}{}

%%---------- Samenvatting -----------------------------------------------------
% De samenvatting in de hoofdtaal van het document

\chapter*{\IfLanguageName{dutch}{Samenvatting}{Abstract}}

\section{Introduction}
This bachelor thesis focuses on the current CI/CD (Continuous Integration/Continuous Delivery) practices at Wolters Kluwer. The objective is to evaluate the environmental impact of these practices and propose optimizations.

\section{Research Question}
The central research question is: "How can CI/CD practices be optimized to minimize their environmental impact?" The goal is to gain insight into the energy consumption and resource utilization of the CI/CD practices, and to make recommendations for improvement.

\section{Methodology}
The research was conducted through a detailed case study at Wolters Kluwer. Data was collected on CPU utilization, memory usage, build metrics, resource usage, and system uptime of the CI/CD pipeline. Subsequently, best practices were implemented and evaluated. A comparison was also made between Linux and Windows agents.

\section{Results}
The main results show that there are significant opportunities to optimize CI/CD practices.

\section{Conclusion}
The research demonstrates that optimizing CI/CD practices not only increases efficiency but also reduces environmental impact. Implementing the proposed best practices can lead to more sustainable software development at Wolters Kluwer. Testing best practices is not easy in an environment that is not fully isolated from other processes. It is important to view the results of this research as a first step towards more sustainable software development.









%---------- Inhoud, lijst figuren, ... -----------------------------------------

\tableofcontents

% In a list of figures, the complete caption will be included. To prevent this,
% ALWAYS add a short description in the caption!
%
%  \caption[short description]{elaborate description}
%
% If you do, only the short description will be used in the list of figures

\listoffigures

% If you included tables and/or source code listings, uncomment the appropriate
% lines.
%\listoftables
%\listoflistings

% Als je een lijst van afkortingen of termen wil toevoegen, dan hoort die
% hier thuis. Gebruik bijvoorbeeld de ``glossaries'' package.
% https://www.overleaf.com/learn/latex/Glossaries

%---------- Kern ---------------------------------------------------------------

\mainmatter{}

% De eerste hoofdstukken van een bachelorproef zijn meestal een inleiding op
% het onderwerp, literatuurstudie en verantwoording methodologie.
% Aarzel niet om een meer beschrijvende titel aan deze hoofdstukken te geven of
% om bijvoorbeeld de inleiding en/of stand van zaken over meerdere hoofdstukken
% te verspreiden!

%%=============================================================================
%% Inleiding
%%=============================================================================

\chapter{\IfLanguageName{dutch}{Inleiding}{Introduction}}%
\label{ch:inleiding}

De inleiding moet de lezer net genoeg informatie verschaffen om het onderwerp te begrijpen en in te zien waarom de onderzoeksvraag de moeite waard is om te onderzoeken. In de inleiding ga je literatuurverwijzingen beperken, zodat de tekst vlot leesbaar blijft. Je kan de inleiding verder onderverdelen in secties als dit de tekst verduidelijkt. Zaken die aan bod kunnen komen in de inleiding~\autocite{Pollefliet2011}:

\begin{itemize}
  \item context, achtergrond
  \item afbakenen van het onderwerp
  \item verantwoording van het onderwerp, methodologie
  \item probleemstelling
  \item onderzoeksdoelstelling
  \item onderzoeksvraag
  \item \ldots
\end{itemize}

\section{\IfLanguageName{dutch}{Probleemstelling}{Problem Statement}}%
\label{sec:probleemstelling}
% Uit je probleemstelling moet duidelijk zijn dat je onderzoek een meerwaarde heeft voor een concrete doelgroep. De doelgroep moet goed gedefinieerd en afgelijnd zijn. Doelgroepen als ``bedrijven,'' ``KMO's'', systeembeheerders, enz.~zijn nog te vaag. Als je een lijstje kan maken van de personen/organisaties die een meerwaarde zullen vinden in deze bachelorproef (dit is eigenlijk je steekproefkader), dan is dat een indicatie dat de doelgroep goed gedefinieerd is. Dit kan een enkel bedrijf zijn of zelfs één persoon (je co-promotor/opdrachtgever).

In the world of software development, the optimization of Continuous Integration and Deployment (CI/CD) processes has become a decisive concern for organizations striving to maintain efficiency and sustainability.
This bachelor's thesis focuses on a concrete problem situation within a company context, with the center of attention on a case study.

The target audience for this research encompasses organisations that are looking to optimize their CI/CD processes and are interested in making a sustainable and energy and cost-efficient choice.
By focusing this research on this specific group, it aims to provide insights and solutions to system administrators who already have knowledge of the CI/CD world.


\section{\IfLanguageName{dutch}{Onderzoeksvraag}{Research question}}%
\label{sec:onderzoeksvraag}
% Wees zo concreet mogelijk bij het formuleren van je onderzoeksvraag. Een onderzoeksvraag is trouwens iets waar nog niemand op dit moment een antwoord heeft (voor zover je kan nagaan). Het opzoeken van bestaande informatie (bv. ``welke tools bestaan er voor deze toepassing?'') is dus geen onderzoeksvraag. Je kan de onderzoeksvraag verder specifiëren in deelvragen. Bv.~als je onderzoek gaat over performantiemetingen, dan 

How can CI/CD processes be optimized to make them more environmentally sustainable while maintaining efficiency
and resource optimization?


\section{\IfLanguageName{dutch}{Onderzoeksdoelstelling}{Research objective}}%
\label{sec:onderzoeksdoelstelling}
% Wat is het beoogde resultaat van je bachelorproef? Wat zijn de criteria voor succes? Beschrijf die zo concreet mogelijk. Gaat het bv.\ om een proof-of-concept, een prototype, een verslag met aanbevelingen, een vergelijkende studie, enz.

Through a thorough analysis and proof-of-concept, the aim is to contribute practical solutions to the identified problem.
The goal is not only to enhance CI/CD efficiency but also to align this process with principles of environmental sustainability.
The success of this research will be gauged by the impact our recommendations have on Wolters Kluwer's CI/CD processes. 
Whether indicated through a proof-of-concept implementation, a comprehensive report featuring actionable recommendations, or other practical outcomes, the benchmark for success lies in positively transforming CI/CD practices toward sustainability.


\section{\IfLanguageName{dutch}{Opzet van deze bachelorproef}{Structure of this bachelor thesis}}%
\label{sec:opzet-bachelorproef}

% Het is gebruikelijk aan het einde van de inleiding een overzicht te
% geven van de opbouw van de rest van de tekst. Deze sectie bevat al een aanzet
% die je kan aanvullen/aanpassen in functie van je eigen tekst.

The rest of this bachelor's thesis is structured as follows:

In Hoofdstuk~\ref{ch:stand-van-zaken} wordt een overzicht gegeven van de stand van zaken binnen het onderzoeksdomein, op basis van een literatuurstudie.

In Hoofdstuk~\ref{ch:methodologie} wordt de methodologie toegelicht en worden de gebruikte onderzoekstechnieken besproken om een antwoord te kunnen formuleren op de onderzoeksvragen.

% TODO: Vul hier aan voor je eigen hoofstukken, één of twee zinnen per hoofdstuk

In Hoofdstuk~\ref{ch:conclusie}, tenslotte, wordt de conclusie gegeven en een antwoord geformuleerd op de onderzoeksvragen. Daarbij wordt ook een aanzet gegeven voor toekomstig onderzoek binnen dit domein.

\chapter{\IfLanguageName{dutch}{Stand van zaken}{State of the art}}%
\label{ch:stand-van-zaken}

% Tip: Begin elk hoofdstuk met een paragraaf inleiding die beschrijft hoe
% dit hoofdstuk past binnen het geheel van de bachelorproef. Geef in het
% bijzonder aan wat de link is met het vorige en volgende hoofdstuk.

% Pas na deze inleidende paragraaf komt de eerste sectiehoofding.

% Dit hoofdstuk bevat je literatuurstudie. De inhoud gaat verder op de inleiding, maar zal het onderwerp van de bachelorproef *diepgaand* uitspitten. De bedoeling is dat de lezer na lezing van dit hoofdstuk helemaal op de hoogte is van de huidige stand van zaken (state-of-the-art) in het onderzoeksdomein. Iemand die niet vertrouwd is met het onderwerp, weet nu voldoende om de rest van het verhaal te kunnen volgen, zonder dat die er nog andere informatie moet over opzoeken \autocite{Pollefliet2011}.

% Je verwijst bij elke bewering die je doet, vakterm die je introduceert, enz.\ naar je bronnen. In \LaTeX{} kan dat met het commando \texttt{$\backslash${textcite\{\}}} of \texttt{$\backslash${autocite\{\}}}. Als argument van het commando geef je de ``sleutel'' van een ``record'' in een bibliografische databank in het Bib\LaTeX{}-formaat (een tekstbestand). Als je expliciet naar de auteur verwijst in de zin (narratieve referentie), gebruik je \texttt{$\backslash${}textcite\{\}}. Soms is de auteursnaam niet expliciet een onderdeel van de zin, dan gebruik je \texttt{$\backslash${}autocite\{\}} (referentie tussen haakjes). Dit gebruik je bv.~bij een citaat, of om in het bijschrift van een overgenomen afbeelding, broncode, tabel, enz. te verwijzen naar de bron. In de volgende paragraaf een voorbeeld van elk.

% \textcite{Knuth1998} schreef een van de standaardwerken over sorteer- en zoekalgoritmen. Experten zijn het erover eens dat cloud computing een interessante opportuniteit vormen, zowel voor gebruikers als voor dienstverleners op vlak van informatietechnologie~\autocite{Creeger2009}.

% Let er ook op: het \texttt{cite}-commando voor de punt, dus binnen de zin. Je verwijst meteen naar een bron in de eerste zin die erop gebaseerd is, dus niet pas op het einde van een paragraaf.


In examining the present landscape of software automation, the breadth of available options mirrors the vastness of an ocean \autocite{King2019}.
The ever-evolving landscape of automation is complex, and if you want to utilize all possible solutions, you may be busy for a long while \autocite{King2019}.
This literature review dives deeper into this vast ocean, aiming to provide an overview of the current landscape op sustainability in CI/CD processes, drawing extensively from professional literature.



\section{The problem in today's software development}
Today the typical development journey mainly focuses on how it will impact an organization instead of addressing how it affects the environment. Next to factors like improving time-to-market, reducing human error, debugging and making resource management more efficient, part of your infrastructure strategy should include ways to reduce your carbon footprint and impact on the planet /autocite{Brode2022}. 
To introduce why it is important to address the problem it helps to take a look at the impact of technology on the climate and planet. \textcite{Brode2022} states that a research in 2022 showed that there are already more than 16 million mobile devices in use. The average emission of one device is around 85kg of CO2 \autocite{Six2023}. Next to the carbon emission, the computing consumes an enormous amount of power. Data centers used around 416 terawatts in 2020 and the Bitcoin network consumed even more than all data centers combined \textcite{Brode2022}. 
 The mobile devices not only emit a lot of emission, they also create a lot of waste. Very little hardware is recycled and electronic waste comprises 70\% of all toxic waste. It is estimated on a total of 40 million tons annually. With the power consumption of data centers they also need cooling. Internet usage uses 3,000 liters of water per person each year and emits 2,000 kg of CO2 \autocite{Brode2022}.
With these numbers only raising each year it is important every company contribute in creating a greener world. Software developers and DevOps teams can participate in this goal by applying best practices. 



\section{Continous Integration and Continous Deployment}
Continuous Integration and Deployment (CI/CD) processess have revolutionized software development by enabling teams to deliver high-quality code at a rapid pace \autocite{Sacolick2024}.
However, as the software industry grapples with the challenges of sustainability, it is imperative to explore how software development, and subsequent, CI/CD practices can be optimized to align with eco-friendly principles.

Continous Integration (CI) is a set of practices that automates the integration of small code changes and check them in to a version control repository \autocite{Sacolick2024}.
By regularly merging code changes into a shared repository, developers can detect and resolve integration issues early in the development process because CI ensures that the each change is tested and verified by a build.
This process does not only save time and money but also encourages developers to commit their code more frequently.
This leads to a better collaboration between team members and a more cohesive and stable codebase.

Continous Delivery (CD) follows up the CI process by automating the release of validated code to a repository \autocite{Hat2023}.
Every stage in this process includes test automation and code quality checks, which should ensure that the code is always in a deployable state.
This approach minimizes the risk of human error and ensures that the code is always ready to be deployed to production. 
Developers can deliver software updates with confidence because the CD process ensures that bad code changes are caught early in the pipeline.

Continuous Deployment (CD) is the next and last step in the CI/CD process.
It reffers to the automation of the release of a developer's change to the production environment after passing through the CI/CD pipeline \autocite{Hat2023}.
CD elminiates the need for manual intervention in the deployment process, which can lead to faster and more reliable deployments.

This CI/CD process can make sure that small changes could go live in a matter of minutes, instead of days or weeks. 
Because there is no manual intervention in the CI/CD process, the deployment relies heavily on well-designed test automation.
This can be a challenge because it needs to cover all possible scenarios, which can be difficult to predict \autocite{Hat2023}.



\section{CI/CD Tools}
The journey of implementing Continuous Integration and Continuous Deployment (CI/CD) begins with the crucial decision of selecting the appropriate tool(s). 
While some tools offer comprehensive solutions covering the entire CI/CD pipeline, others specialize in specific aspects of the process \autocite{Hat2023}. 
When making this decision, it is important to assess various factors, including the tool's feature set, usability, scalability, and cost \autocite{Synopsys}. 
By carefully considering these aspects, teams can ensure that the chosen tool aligns with their unique requirements and organizational goals.
In the following sections, we take a look at some of the most widely used CI/CD tools, providing insights into their functionalities and suitability for different use cases.


\subsection{Jenkins}
Jenkins is an open-source automation server that is widely used for CI/CD processes. It provides a variety of plugins to support building and deploying applications. 
Jenkins is highly extensible and can be easily integrated with other tools and services. 
It is a popular choice for organizations looking to implement CI/CD pipelines due to its flexibility and robust feature set \autocite{Hat2023}.


\subsection{GitLab}
GitLab is a web-based Git repository manager CI/CD pipeline features using an open-source license.
It provides a complete DevOps platform that enables teams to manage their code, plan, build, verify, package, release, configure, and monitor applications.


\subsection{Jetbrains TeamCity}
Teramcity is a Java-based CI/CD server that is developed by JetBrains. It offers support for various programs like Docker, Jira, etc.
Jetbrains Teamcity is the CI/CD tool that is used at Wolters Kluwer, the company where a part of this research is conducted.



\section{Sustainability of CI/CD}
In the wake of escalating environmental concerns, the newsletter of \textcite{Corewave2023} underscores the imperative for sustainable practices in software and app development.
This section will discuss some best practices for sustainable DevOps, which can help organizations reduce their carbon footprint and contribute to a greener future.
Beyond the environmental benefits, embracing sustainable development offers business cost savings, economic advantages, improved user experiences, and an enhanced brand reputation \autocite{Corewave2023}.


\subsection{Code optimization}
Writing efficient code is a fundamental aspect of sustainable software development. 
The implementation of code is a resource and energy-intensive process, and optimizing code can help reduce the energy consumption of applications.
By prioritizing clean code practices, that are summed up in the following subsections, and avoiding unnecessary code, developers can create more sustainable applications that consume fewer resources \autocite{Corewave2023}.

\subsubsection{Identify bottlenecks}
Every line of code requires some CPU time. That is why it is important to measure and analyze the performance of your code. This way you can identify slow and efficient parts and work on them to optimize your process \autocite{Ojeda}.

\subsubsection{Choose the right data structures and algorithms}
In this part it is recommended to choose the data structure and algorithm that suits best for your problem. For example using a hash table to storing and retrieving data quickly. Avoid using unnecessary structures such as creating copies of data \autocite{Ojeda}.

\subsubsection{Use built-in or standard libraries}
Another option to optimize your code is to use the built-in libraries that come with your programming language. These libraries often provide optimized functions such as the math library in Python to perform mathematical functions like calculating a square root \autocite{Ojeda}.


\subsection{Infrastructure optimization} 
\textcite{Krivec2023} states that more than 40\% of enterprises use cloud automation. This can optimize costs, time, and resources. By leveraging cloud services, organizations can reduce the energy consumption of their infrastructure and improve the efficiency of their CI/CD pipelines.
But your choice of cloud provider can also have an impact on the sustainability of your CI/CD process. Some prioritize energy-efficient data centers and adopting practices such as serverless computing to optimize their use \autocite{Festus2024}


\subsection{User education}
Educating users about the environmental impact of software development can help raise awareness and empowers them to make choices that align with sustainable goals \autocite{Festus2024}. 
It could promote writing more efficient code and applying best practices.


\subsection{Lifecycle Management and Decommissioning}
Sustainable automation extends to the entire lifecycle of infrastructure components. Another big part is the decommissioning and recycling of hardware. This includes responsible disposal or retired equipment and minimizing electronic waste, contributing to the overall sustainability of technology operations \autocite{Festus2024}.


\subsection{Infrastructure as Code}
Infrastrcture as Code (IaC) is the managing and provisioning of infrastructure through code instead of through manual processes \autocite{RedHat2022}. IaC allows DevOps teams to define and manage infrastructure using code, enabling versioning, collaboration, and the ability to reproduce environments. This enhances the efficiency of infrastructure management and reduces the likelihood of misconfigurations that could lead to unnecessary resource usage \autocite{Festus2024}


\subsection{Clean-up CI/CD environment}
Like mentioned this bachelor thesis is based on the internship at Wolters Kluwer. They have more than 27 thousand jobs on their environment and it uses unnecessary memory. 
Jobs that are not used anymore take up memory and should be archived. But also jobs with wrong configurations should be tackled.
This is again an example of user education. The developers should pay attention to this and archive their jobs when they are not used anymore.


\subsection{Code design}
Code developers should aim to extend the life cycle of their products by designing them to be easily maintained and updated but also to reduce the need for frequent updates \autocite{Zudu2024}.
But also adding features such as dark mode, automatic screen brightness adjustment and efficient data caching can help reduce the energy consumption of applications.


\subsection{Containerization}
A container is lightweight software with components that bundle and package the application, its dependencies and its configuration in an image. The image operates within isolated user environments on conventional operating systems.
The scaling of automated build agents, which is explained later in this state-of-art, is a nice example of the efficiency of containerization. But this is not all it has to offer.

\subsubsection{Automated build and testing}
Containers could be used to bundle the correct tool, version and other execution assets in a package. This makes it easy to build a given application with a set of tools and scripts as the package is ready to run. This container can be controlled by several providers such as Kubernetes.
Not only the build can be automated, the testing tools and scripts can be packaged in separate containers.

\subsubsection{Clean environment}
A well packaged containers could not contain impurities that could affect the execution of a build. A build can’t affect a container and a same build will always get you the same result because of the creation of a new image for every build. 
It is also possible to create a production grade container that reflects a clone of your production environment. This way the principle of continuous delivery is supported because code can be tested in a production environment before pushing it to production.
By containerizing your resources you avoid the sharing of resources which could result in a slower process. This way no resources go to waste and there are always enough resources at any moment just like the automated scaling of the build agents
Because every run is in a separate container it is isolated from other resources. This protects the current and other application from data breaches during the CI/CD process.

\subsubsection{Adoption and scaling with containers}
With the use of containers team can easily adopt different tools and technologies with various customizations. Container based CI/CD solutions can be easily migrated to the cloud. In case your enterprise would want to switch to another CI/CD provider it would be a lot easier to move your containers instead of separate jobs. 
It also makes scaling a lot easier. Containers can scale in a few seconds, while traditional infrastructures could take minutes to hours. You can scale them both vertically (add more power to it such as adding CPU power) and horizontally (add more containers) \autocite{Samant2021}. The only thing you need is a container orchestrator which makes the managing of your containers easier as well.


\subsection{Monitoring in CI/CD}
A step towards an efficient and sustainable CI/CD pipeline is the monitoring of it. It gives you a proper overview of the performance of your pipeline. 
•	Long-term trends: You can track the count of builds, but also the workload on specific moments of a day. This way you can see the need to scale your infrastructure
•	Over-time comparison: You can see the speed your builds run compared to the speed of last week or longer
•	Alerting: if something is broken you can track the exact time of the break and you will be able to rollback builds that couldn’t complete
•	Ad-hoc retrospective analysis: by proper analyzing different metrics you are able to see which events relate and impact each other
Monitoring the CI/CD pipeline is an important aspect in this research and will give an overview on the methods that improve sustainability. The used metrics for the research can be found in the methodology.

\subsubsection{Prometheus}
There are some tools to observe the performance of your environment but the most convenient one is Prometheus. This is also the tool that will be used to support this research. You will first need to configure Prometheus with your pipeline. Then it is pretty easy to build a Grafana style dashboard. Once you have made your dashboard it is time to create some panels. It is important to determine the metrics to track. These can vary from CPU usage of your server to a container status to JVM memory usage. Grafana offers many visualizations to present the data.
\cite{MetricFire2023}



\section{Case Studies in Sustainable Development}
In this section we will take a look at some case studies of companies that have implemented sustainable practices in their software development \autocite{Corewave2023}.


\subsection{Google}
Google is a leader in purchasing renewable energy and has been carbon neutral since 2007 \autocite{Pichai2020}. They committed to powering their operations with 100\% renewable energy by 2030 \autocite{PetersonCorio2022}.
Their Android Go program focuses on developing apps that are optimized for low-end devices, which can help reduce energy consumption \autocite{Corewave2023}.


\subsection{Apple}
Apple has made significant strides in reducing its carbon footprint by transitioning to renewable energy sources. 
They encourage developers to create energy-efficient apps by providing tools and resources to optimize their code \autocite{Corewave2023}.


\subsection{Ecosia}
Ecosia, a search engine that plants trees with its ad revenue, is a prime example of a sustainable software company. 
Users can support forest restoration projects by just using the search engine, which has planted over 200 million trees to date \autocite{Corewave2023}.



\section{Build agents}
In almost all CI/CD pipelines, the build agent is a crucial component. An agent is a service that runs the build and deployment processes \autocite{packt}.
The agent can be run on a physical machine, a virtual machine, or a container. 
You would need at least one agent to run your build and deployment processes, but you can also run multiple agents to parallelize your build and deployment processes.
When an agent is busy running a build or deployment process, it is not available to run another process.
This can lead to bottlenecks in your CI/CD pipeline, especially when you have a lot of builds and deployments to run.
In the following subsections we will take a look at some strategies to optimize build agents and make them more sustainable.


\subsection{Build agent scaling}
To avoid bottlenecks and long queue times, you can decide to create more build agents.
You can choose to use Microsoft-hosted agents, which you would not have to manage yourself, but this could get expensive.
You could also choose to create your own build agents, which you would have to manage yourself, which could be very time-consuming \autocite{Keiholz2023}.

Luckily there is a solution to dynamically create build agents when you need them and destroy them when you don't need them.
This can all be done automatically by using Azure DevOps and Azure Virtual Machines. You can create a set of pool rules so Azure knows when to create a new build agent and when to destroy it.
But there is one last issue to solve, after automating the creation and destruction of build agents you would still have to install all the necessary software on the build agent.
This can be solved by creating a custom VM image that already has all the necessary software installed \autocite{Keiholz2023}.
Not only will this move your CI/CD infrastructure to the cloud, but it will also save a lot of power by avoiding idle build agents.
The Azure Virtual Machines cost a lot of money so this process minimizes the cost of running build agents and saves up time creating a new build agent.



\section{Shifting to Linux}
Another way to optimize your build agents is to shift the operating system to Linux. 
The switch from Windows to Linux not only offers better cost-effectiveness and operational effiency but also underscores a significant contribution to environmental sustainability \autocite{Germain2017}.
\textcite{Germain2017} states a success story of Ethan T. Schmidt, chief technology officer at GymBull.com, who shifted to Linux and never looked back.
Although there was some initial apprehension from his non-developer staff about leaving Windows behind, the transition was almost transparent. 
The Ubuntu interface of today is virtually the same as Windows, and it helps that they have Linux troubleshooters to help out anyone who needs it. 

The following are some of the reasons why Linux is a better choice as operating system, not only for build agents but also for other purposes.


\subsection{Open source}
Linux is open source, which means that the source code is freely available to the public. 
This makes Linux more secure, stable and reliable than Windows \autocite{Singh2023}.


\subsection{Customization}
Linux is highly customizable, which means that you can tailor it to your specific needs. 
Users are able to choose from a wide range of distributions, each with its own set of features and capabilities \autocite{Singh2023}.


\subsection{Security}
Linux is known for its robust security features, which make it less vulnerable to malware and other cyber threats.
It is less targeted by hackers than Windows, and due to their open source nature, vulnerabilities are quickly identified and patched \autocite{Singh2023}.


\subsection{Cost}
Linux is free to use, which can result in significant cost savings for organizations \autocite{Singh2023}. 
You don't have to pay for licenses, and you can run Linux on older hardware, which can extend the life of your machines.


\subsection{Performance}
Linux is known for its superior performance, which can lead to faster build times and improved productivity.
Wether you are just using your personal computer or a build agent in the cloud, Linux is a better choice for performance \autocite{Singh2023}.
Better performance means that you can run more builds and deployments in less time, which can help you deliver software faster.


%%=============================================================================
%% Methodologie
%%=============================================================================

\chapter{\IfLanguageName{dutch}{Methodologie}{Methodology}}%
\label{ch:methodologie}

%% TODO: In dit hoofstuk geef je een korte toelichting over hoe je te werk bent
%% gegaan. Verdeel je onderzoek in grote fasen, en licht in elke fase toe wat
%% de doelstelling was, welke deliverables daar uit gekomen zijn, en welke
%% onderzoeksmethoden je daarbij toegepast hebt. Verantwoord waarom je
%% op deze manier te werk gegaan bent.
%% 
%% Voorbeelden van zulke fasen zijn: literatuurstudie, opstellen van een
%% requirements-analyse, opstellen long-list (bij vergelijkende studie),
%% selectie van geschikte tools (bij vergelijkende studie, "short-list"),
%% opzetten testopstelling/PoC, uitvoeren testen en verzamelen
%% van resultaten, analyse van resultaten, ...
%%
%% !!!!! LET OP !!!!!
%%
%% Het is uitdrukkelijk NIET de bedoeling dat je het grootste deel van de corpus
%% van je bachelorproef in dit hoofstuk verwerkt! Dit hoofdstuk is eerder een
%% kort overzicht van je plan van aanpak.
%%
%% Maak voor elke fase (behalve het literatuuronderzoek) een NIEUW HOOFDSTUK aan
%% en geef het een gepaste titel.

\section{Phase 1: Literary review}
The research will be divided into 4 phases, the first of which is a literature review. This will be performed to examine existing approaches and gain deeper insights into the evolution of CI/CD practices.

\section{Phase 2: Metrics and Tools Analysis}
This phase will delve into the design of the research, focusing on the methodology and the approach taken to address the research questions. The chapter will also provide an overview of the research design, the research methods, and the data collection techniques used to address the research questions.

\section{Phase 3: Case Study}
The subsequent phase of the bachelor thesis involves a case study at Wolters Kluwer to evaluate the environmental impact of the current CI/CD practices. This case study will provide insight into the environmental impact by analyzing the energy consumption, carbon footprint, and resource utilization of the CI/CD practices at Wolters Kluwer. Additionally, the case study will evaluate the efficiency and scalability of these CI/CD practices.

\section{Phase 4: Comparative Study}
The fourth phase of the research will involve a comparative study between Windows and Linux agents. This phase will compare the energy-efficiency and performance of Windows and Linux agents in a TeamCity instance. 

\section{Phase 5: Conclusion}
The findings from the comparative study of each shortlisted approach will be summarized to draw conclusions regarding the most effective and practical, sustainable CI/CD practices. This summarization will enable the provision of recommendations for organizations, such as my internship company, that are seeking to implement sustainable CI/CD processes. These recommendations will be based on the identified strengths and weaknesses of each approach, facilitating informed decision-making and effective implementation strategies for sustainable CI/CD practices.



% Voeg hier je eigen hoofdstukken toe die de ``corpus'' van je bachelorproef
% vormen. De structuur en titels hangen af van je eigen onderzoek. Je kan bv.
% elke fase in je onderzoek in een apart hoofdstuk bespreken.

%\input{...}
%\input{...}
%...

%%=============================================================================
%% Conclusie
%%=============================================================================

\chapter{Discussion}%
\label{ch:discussion}

% TODO: Trek een duidelijke conclusie, in de vorm van een antwoord op de
% onderzoeksvra(a)g(en). Wat was jouw bijdrage aan het onderzoeksdomein en
% hoe biedt dit meerwaarde aan het vakgebied/doelgroep? 
% Reflecteer kritisch over het resultaat. In Engelse teksten wordt deze sectie
% ``Discussion'' genoemd. Had je deze uitkomst verwacht? Zijn er zaken die nog
% niet duidelijk zijn?
% Heeft het onderzoek geleid tot nieuwe vragen die uitnodigen tot verder 
%onderzoek?

This research focused on evaluating and optimizing the CI/CD practices at Wolters Kluwer to minimize their environmental impact. The central research question was: "How can CI/CD practices be optimized to minimize their environmental impact?"

Our study contributes to the growing body of knowledge on sustainable software development practices. By identifying and implementing best practices for CI/CD pipelines, we provide actionable insights that can help other organizations reduce their environmental footprint. Our findings emphasize the importance of resource efficiency in the software development lifecycle and highlight practical steps that can be taken to achieve this.

The results of our study offer significant value to IT professionals and organizations striving to adopt more sustainable practices. The recommendations provided can help improve operational efficiency, reduce energy consumption, and promote sustainable development practices across the industry.

Reflecting on the outcomes, the results were less clear than expected. While the implementation of best practices did lead to some improvements in resource utilization and efficiency, the overall impact on the environment was less significant than anticipated. This suggests that further research and optimization are needed to achieve more substantial reductions in environmental impact. The reason behind this could be other processes influencing the CI/CD pipeline that were not considered in this study.

The comparative analysis between Linux and Windows agents revealed some interesting insights into the performance and energy efficiency of different operating systems. The results showed that Windows agents were more energy-efficient than Linux agents, with lower CPU and memory usage. This suggests that organizations may benefit from using Windows agents in their CI/CD pipelines to reduce energy consumption and improve resource utilization.



%---------- Bijlagen -----------------------------------------------------------

\appendix

\chapter{Onderzoeksvoorstel}

Het onderwerp van deze bachelorproef is gebaseerd op een onderzoeksvoorstel dat vooraf werd beoordeeld door de promotor. Dat voorstel is opgenomen in deze bijlage.

%% TODO: 
%\section*{Samenvatting}

% Kopieer en plak hier de samenvatting (abstract) van je onderzoeksvoorstel.

With the current global warming and high energy prices, companies are knocking at the door to save on resources. 
This study conducts a comparative analysis, researching existing CI/CD frameworks, and explores strategies for enhancing their eco-friendliness. 
Focusing on the CI/CD infrastructure at Wolters Kluwer, the research evaluates the environmental impact of current processes and proposes innovative approaches to make CI/CD more sustainable. 
Through a proof of concept, the thesis implements and assesses these strategies, aiming to contribute insights into the practical application of green computing principles in the software development life cycle. 
The result will not only offer valuable perspectives for the IT industry but also underline the importance of integrating environmental considerations into the core of CI/CD practices for a more sustainable technological future.


% Verwijzing naar het bestand met de inhoud van het onderzoeksvoorstel
%---------- Inleiding ---------------------------------------------------------

\section{Introduction}%
\label{sec:introduction}
In the world of software development, the optimization of Continuous Integration and Deployment (CI/CD) processes has become a decisive concern for organizations striving to maintain efficiency and sustainability. This bachelor's thesis focuses on a concrete problem situation within a company context, with the center of attention on a case study.

The target audience for this research encompasses IT professionals, especially those who are engaged in CI/CD processes. By focusing this research on this specific group, it aims to provide insights and solutions to system administrators who already have knowledge of the CI/CD world.

The case study guiding this research is Wolters Kluwer Financial Services FRR. It's an organization grappling with challenges related to resource optimization, efficiency, and environmental impact within their CI/CD infrastructure, TeamCity. The company is looking for more sustainable and resource-friendly solutions due to bad configurations in the past.

The central research question steering this thesis is: "How can CI/CD processes be optimized to make them more environmentally sustainable while maintaining efficiency
and resource optimization?"

Through a thorough analysis and proof-of-concept, the aim is to contribute practical solutions to the identified problem. The goal is not only to enhance CI/CD efficiency but also to align this process with principles of environmental sustainability.

The success of this research will be gauged by the impact our recommendations have on Wolters Kluwer's CI/CD processes. Whether indicated through a proof-of-concept implementation, a comprehensive report featuring actionable recommendations, or other practical outcomes, the benchmark for success lies in positively transforming CI/CD practices toward sustainability.
%---------- Stand van zaken ---------------------------------------------------

\section{State-of-the-art}%
\label{sec:state-of-the-art}

In examining the present landscape of software automation, the breadth of available options mirrors the vastness of an ocean. \autocite{King2019} The ever-evolving landscape of automation is complex, and if you want to utilize all possible solutions, you may be busy for a long while. \autocite{King2019} This literature review aims to provide an overview of the current state of CI/CD processes, drawing extensively from professional literature.

\subsection{CI/CD Infrastructure and Efficiency}

It all initiates from the outset: which tool shall be utilized? It is important to consider multiple factors, like features relevant to your needs, the ease of use, scalability and cost of the tool. Furthermore, it is important to break the pipeline down into multiple phases, this can dismantle bottlenecks and accelerates the delivery process. The article of \textcite{Dahunsi2023} further delves into the significance of automated testing and Quality Assurance, paralleling builds for scalability, implementing monitoring and feedback loops, conducting security checks and embracing a culture of continuous improvement using the Deming Cycle. The journey to an optimal CI/CD pipeline is a perpetual quest in the ever-evolving landscape of DevOps.

\subsection{Sustainability in CI/CD}

In the wake of escalating environmental concerns, \textcite{Corewave2023}'s newsletter underscores the imperative for sustainable practices in software and app development. Highlighting eco-friendly strategies, the piece emphasizes code optimization, cloud computing efficiency, energy-efficient algorithms, and user education as key contributors to a more sustainable approach. Beyond environmental benefits, embracing sustainable development offers businesses cost savings, economic advantages, improved user experiences, and an enhanced brand reputation. Case studies featuring tech giants like Google, Apple, and Ecosia illustrate successful implementations of sustainable initiatives, reinforcing the notion that integrating eco-friendly methods into software and application development is not just a trend but a crucial step towards a more resilient and sustainable future.

\subsection{Shifting to Linux}

The shift from Windows agents to Linux Virtual machines not only offers better cost-effectiveness and operational efficiency but also underscores a significant contribution to environmental sustainability. \autocite{Germain2017} Linux's open-source nature, coupled with its lightweight yet powerful design, promotes efficient use of resources and energy. Code optimization practices in Linux development play a significant role in reducing energy consumption, thereby promoting environmental consciousness. Linux's compatibility with cloud computing also contributes to energy efficiency by enabling the selection of providers that prioritize eco-friendly data centers and practices like serverless computing. Moreover, Linux's emphasis on energy-efficient algorithms and user education reinforces its reputation as a platform with a reduced environmental footprint. Opting for Linux Virtual Machines over Windows agents not only benefits businesses in terms of operational costs but, also actively supports sustainability initiatives, thereby positively impacting the environment.

Ethan T. Schmidt, chief technology officer at GymBull.com, shifted to Linux and never looked back. Although there was some initial apprehension from his non-developer staff about leaving Windows behind, the transition was almost transparent. The Ubuntu interface of today is virtually the same as Windows, and it helps that they have Linux troubleshooters to help out anyone who needs it. \autocite{Germain2017}

% Voor literatuurverwijzingen zijn er twee belangrijke commando's:
% \autocite{KEY} => (Auteur, jaartal) Gebruik dit als de naam van de auteur
%   geen onderdeel is van de zin.
% \textcite{KEY} => Auteur (jaartal)  Gebruik dit als de auteursnaam wel een
%   functie heeft in de zin (bv. ``Uit onderzoek door Doll & Hill (1954) bleek
%   ...'')


%---------- Methodologie ------------------------------------------------------
\section{Methodology}%
\label{sec:methodologie}

\subsection{Phase 1: Literary review}
The initial phase of my research involves conducting a comprehensive review of academic literature and web articles pertinent to sustainability, CI/CD processes, and their eco-friendly practices. This will be achieved through targeted searches using relevant terms and keywords. Upon completion of this study, my aim is to gain deeper insights into the evolution of CI/CD processes and their alignment with sustainability principles. Specifically, I intend to acquire a better understanding of the various strategies, tools, and approaches utilized for integrating sustainability into CI/CD pipelines.

\subsection{Phase 2: Shortlist}
Following the literature review, the next step involves validating the identified information through practical observation at Wolters Kluwer. This entails examining their CI/CD processes and conducting interviews with colleagues to gather qualitative insights into sustainable CI/CD practices. By leveraging this firsthand information, I aim to compile a shortlist of the most promising sustainability practices, frameworks, and tools based on their practical applicability. In making this shortlist, factors such as scalability and ease of adoption, as mentioned in the state-of-the-art literature, will be carefully considered.

\subsection{Phase 3: Metrics and Tools Analysis}
A set of sustainability metrics and key performance indicators relevant to CI/CD processes will be developed from the evaluations conducted. Existing tools and software designed to support more sustainable CI/CD processes will be evaluated based on these metrics, enabling the identification of solutions that align closely with the defined sustainability criteria. This evaluation process will facilitate informed decision-making regarding the adoption and implementation of sustainable CI/CD practices.

\subsection{Phase 4: Proof-of-Concept}
The subsequent phase of the bachelor thesis involves crafting a comprehensive plan and a Proof-of-Concept that delineates the objectives, scope, and metrics to be assessed. Criteria for success will be defined to evaluate the effectiveness of each shortlisted approach. Subsequently, the Proof-of-Concept will be implemented within Wolters Kluwer's controlled CI/CD environment. This implementation phase will enable the measurement and collection of data pertaining to environmental impact, efficiency, and other relevant metrics. Concurrently, this initiative will aid Wolters Kluwer in advancing towards more sustainable and efficient CI/CD practices.

\subsection{Phase 5: Conclusion}
The findings from the comparative study of each shortlisted approach will be summarized to draw conclusions regarding the most effective and practical, sustainable CI/CD practices. This summarization will enable the provision of recommendations for organizations, such as my internship company, that are seeking to implement sustainable CI/CD processes. These recommendations will be based on the identified strengths and weaknesses of each approach, facilitating informed decision-making and effective implementation strategies for sustainable CI/CD practices.

\begin{figure}
    \centering
    \includegraphics[width=0.8\linewidth]
    {graphics/Gantt.png}
    \caption{\label{fig:protocol}Gannt-diagram of the methodology}
\end{figure}

%---------- Verwachte resultaten ----------------------------------------------
\section{Expected results, conclusion}%
\label{sec:verwachte_resultaten}

The conclusion of the comparative study is expected to offer several key insights. Firstly, the identification of optimal sustainability practices for CI/CD processes sheds light on why certain approaches outperform others in reducing environmental impact and resource consumption. The findings will also address the scalability of each shortlisted sustainability approach, providing insights into their suitability for different organizational scales.

Additionally, the study will contribute recommendations for organizations aiming to implement sustainable CI/CD processes. These recommendations will encompass best practices, potential challenges, and considerations for successful adoption. The delicate balance between CI/CD efficiency and sustainability goals will be acknowledged, offering guidance on optimizing both aspects. Ultimately, the study's contributions aim to advance the field of sustainable software development, providing valuable insights for practitioners, researchers, and organizations seeking to align their CI/CD processes with environmental sustainability.











%%---------- Andere bijlagen --------------------------------------------------
% TODO: Voeg hier eventuele andere bijlagen toe. Bv. als je deze BP voor de
% tweede keer indient, een overzicht van de verbeteringen t.o.v. het origineel.
%\input{...}

%%---------- Backmatter, referentielijst ---------------------------------------

\backmatter{}

\setlength\bibitemsep{2pt} %% Add Some space between the bibliograpy entries
\printbibliography[heading=bibintoc]

\end{document}
