%%=============================================================================
%% Samenvatting
%%=============================================================================

% TODO: De "abstract" of samenvatting is een kernachtige (~ 1 blz. voor een
% thesis) synthese van het document.
%
% Een goede abstract biedt een kernachtig antwoord op volgende vragen:
%
% 1. Waarover gaat de bachelorproef?
% 2. Waarom heb je er over geschreven?
% 3. Hoe heb je het onderzoek uitgevoerd?
% 4. Wat waren de resultaten? Wat blijkt uit je onderzoek?
% 5. Wat betekenen je resultaten? Wat is de relevantie voor het werkveld?
%
% Daarom bestaat een abstract uit volgende componenten:
%
% - inleiding + kaderen thema
% - probleemstelling
% - (centrale) onderzoeksvraag
% - onderzoeksdoelstelling
% - methodologie
% - resultaten (beperk tot de belangrijkste, relevant voor de onderzoeksvraag)
% - conclusies, aanbevelingen, beperkingen
%
% LET OP! Een samenvatting is GEEN voorwoord!

%%---------- Nederlandse samenvatting -----------------------------------------
%
% TODO: Als je je bachelorproef in het Engels schrijft, moet je eerst een
% Nederlandse samenvatting invoegen. Haal daarvoor onderstaande code uit
% commentaar.
% Wie zijn bachelorproef in het Nederlands schrijft, kan dit negeren, de inhoud
% wordt niet in het document ingevoegd.

\IfLanguageName{english}{%
\selectlanguage{dutch}
\chapter*{Samenvatting}

\section{Inleiding}
Deze bachelorproef richt zich op de huidige CI/CD-praktijken (Continuous Integration/Continuous Delivery) bij Wolters Kluwer. Het doel is om de milieu-impact van deze praktijken te evalueren en optimalisaties voor te stellen.

\section{Onderzoeksvraag}
De centrale onderzoeksvraag luidt: "Hoe kunnen de CI/CD-praktijken worden geoptimaliseerd om hun milieu-impact te minimaliseren?" De doelstelling is om inzicht te krijgen in het energieverbruik en grondstofgebruik van de CI/CD-praktijken, en aanbevelingen te doen voor verbetering.

\section{Methodologie}
Het onderzoek werd uitgevoerd door een gedetailleerde casestudy bij Wolters Kluwer. Er werd data verzameld over CPU-gebruik, geheugenverbruik, bouwstatistieken, grondstofgebruik en systeem-uptime van de CI/CD-pipeline. Vervolgens werden de beste praktijken geïmplementeerd en geëvalueerd.

\section{Resultaten}
De belangrijkste resultaten tonen aan dat er heel wat mogelijkheden zijn om de CI/CD-praktijken te optimaliseren. 

\section{Conclusie}
Het onderzoek toont aan dat het optimaliseren van CI/CD-praktijken niet alleen de efficiëntie verhoogt, maar ook de milieu-impact vermindert. Het implementeren van de voorgestelde beste praktijken kan leiden tot duurzamere softwareontwikkeling bij Wolters Kluwer. Het testen van best practices is niet gemakkelijk op een omgeving die niet volledig is geïsoleerd van andere processen. Het is belangrijk om de resultaten van dit onderzoek te zien als een eerste stap in de richting van een duurzamere softwareontwikkeling.


\selectlanguage{english}
}{}

%%---------- Samenvatting -----------------------------------------------------
% De samenvatting in de hoofdtaal van het document

\chapter*{\IfLanguageName{dutch}{Samenvatting}{Abstract}}

\section{Introduction}
This bachelor thesis focuses on the current CI/CD (Continuous Integration/Continuous Delivery) practices at Wolters Kluwer. The objective is to evaluate the environmental impact of these practices and propose optimizations.

\section{Research Question}
The central research question is: "How can CI/CD practices be optimized to minimize their environmental impact?" The goal is to gain insight into the energy consumption and resource utilization of the CI/CD practices, and to make recommendations for improvement.

\section{Methodology}
The research was conducted through a detailed case study at Wolters Kluwer. Data was collected on CPU utilization, memory usage, build metrics, resource usage, and system uptime of the CI/CD pipeline. Subsequently, best practices were implemented and evaluated.

\section{Results}
The main results show that there are significant opportunities to optimize CI/CD practices.

\section{Conclusion}
The research demonstrates that optimizing CI/CD practices not only increases efficiency but also reduces environmental impact. Implementing the proposed best practices can lead to more sustainable software development at Wolters Kluwer. Testing best practices is not easy in an environment that is not fully isolated from other processes. It is important to view the results of this research as a first step towards more sustainable software development.







