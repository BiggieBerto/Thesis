%%=============================================================================
%% Inleiding
%%=============================================================================

\chapter{\IfLanguageName{dutch}{Inleiding}{Introduction}}%
\label{ch:inleiding}

\section{\IfLanguageName{dutch}{Probleemstelling}{Problem Statement}}%
\label{sec:probleemstelling}
% Uit je probleemstelling moet duidelijk zijn dat je onderzoek een meerwaarde heeft voor een concrete doelgroep. De doelgroep moet goed gedefinieerd en afgelijnd zijn. Doelgroepen als ``bedrijven,'' ``KMO's'', systeembeheerders, enz.~zijn nog te vaag. Als je een lijstje kan maken van de personen/organisaties die een meerwaarde zullen vinden in deze bachelorproef (dit is eigenlijk je steekproefkader), dan is dat een indicatie dat de doelgroep goed gedefinieerd is. Dit kan een enkel bedrijf zijn of zelfs één persoon (je co-promotor/opdrachtgever).

In the world of software development, the optimization of Continuous Integration and Deployment (CI/CD) processes has become a decisive concern for organizations striving to maintain efficiency and sustainability.
This bachelor's thesis focuses on a concrete problem situation within a company context, with the center of attention on a case study.

The target audience for this research encompasses organisations that are looking to optimize their CI/CD processes and are interested in making a sustainable and energy and cost-efficient choice.
By focusing this research on this specific group, it aims to provide insights and solutions to system administrators who already have knowledge of the CI/CD world.


\section{\IfLanguageName{dutch}{Onderzoeksvraag}{Research question}}%
\label{sec:onderzoeksvraag}
% Wees zo concreet mogelijk bij het formuleren van je onderzoeksvraag. Een onderzoeksvraag is trouwens iets waar nog niemand op dit moment een antwoord heeft (voor zover je kan nagaan). Het opzoeken van bestaande informatie (bv. ``welke tools bestaan er voor deze toepassing?'') is dus geen onderzoeksvraag. Je kan de onderzoeksvraag verder specifiëren in deelvragen. Bv.~als je onderzoek gaat over performantiemetingen, dan 

How can CI/CD processes be optimized to make them more environmentally sustainable while maintaining efficiency
and resource optimization?


\section{\IfLanguageName{dutch}{Onderzoeksdoelstelling}{Research objective}}%
\label{sec:onderzoeksdoelstelling}
% Wat is het beoogde resultaat van je bachelorproef? Wat zijn de criteria voor succes? Beschrijf die zo concreet mogelijk. Gaat het bv.\ om een proof-of-concept, een prototype, een verslag met aanbevelingen, een vergelijkende studie, enz.

Through a thorough analysis and case study, the aim is to contribute practical solutions to the identified problem.
The goal is not only to enhance CI/CD efficiency but also to align this process with principles of environmental sustainability.
The success of this research will be gauged by the impact our recommendations have on Wolters Kluwer's CI/CD processes. 
Whether indicated through a proof-of-concept implementation, a comprehensive report featuring actionable recommendations, or other practical outcomes, the benchmark for success lies in positively transforming CI/CD practices toward sustainability.


\section{\IfLanguageName{dutch}{Opzet van deze bachelorproef}{Structure of this bachelor thesis}}%
\label{sec:opzet-bachelorproef}

% Het is gebruikelijk aan het einde van de inleiding een overzicht te
% geven van de opbouw van de rest van de tekst. Deze sectie bevat al een aanzet
% die je kan aanvullen/aanpassen in functie van je eigen tekst.

The rest of this bachelor's thesis is structured as follows:

In Chapter~\ref{ch:stand-van-zaken}, an overview is given of the state of affairs within the research domain, based on a literature review.

In Chapter~\ref{ch:methodologie}, the methodology is explained, and the research techniques used to answer the research questions are discussed.

In Chapter~\ref{ch:metrics}, the metrics used to evaluate the CI/CD processes are discussed. The used tools to measure these metrics are also explained.

In Chapter~\ref{ch:casestudy}, the case study at Wolters Kluwer is discussed, focusing on the implementation of the CI/CD best practices listed and the results obtained.

In Chapter~\ref{ch:comparison}, a comparison is made between the Linux and Windows agents, focusing on the performance and energy efficiency of the agents.

In Chapter~\ref{ch:discussion}, the conclusion is given, and an answer is formulated to the research questions. An outline is also given for future research within this domain.