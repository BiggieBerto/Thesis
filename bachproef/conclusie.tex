%%=============================================================================
%% Conclusie
%%=============================================================================

\chapter{Discussion}%
\label{ch:discussion}

% TODO: Trek een duidelijke conclusie, in de vorm van een antwoord op de
% onderzoeksvra(a)g(en). Wat was jouw bijdrage aan het onderzoeksdomein en
% hoe biedt dit meerwaarde aan het vakgebied/doelgroep? 
% Reflecteer kritisch over het resultaat. In Engelse teksten wordt deze sectie
% ``Discussion'' genoemd. Had je deze uitkomst verwacht? Zijn er zaken die nog
% niet duidelijk zijn?
% Heeft het onderzoek geleid tot nieuwe vragen die uitnodigen tot verder 
%onderzoek?

This research focused on evaluating and optimizing the CI/CD practices at Wolters Kluwer to minimize their environmental impact. The central research question was: "How can CI/CD practices be optimized to minimize their environmental impact?"

Our study contributes to the growing body of knowledge on sustainable software development practices. By identifying and implementing best practices for CI/CD pipelines, we provide actionable insights that can help other organizations reduce their environmental footprint. Our findings emphasize the importance of resource efficiency in the software development lifecycle and highlight practical steps that can be taken to achieve this.

The results of our study offer significant value to IT professionals and organizations striving to adopt more sustainable practices. The recommendations provided can help improve operational efficiency, reduce energy consumption, and promote sustainable development practices across the industry.

Reflecting on the outcomes, the results were less clear than expected. While the implementation of best practices did lead to some improvements in resource utilization and efficiency, the overall impact on the environment was less significant than anticipated. This suggests that further research and optimization are needed to achieve more substantial reductions in environmental impact. The reason behind this could be other processes influencing the CI/CD pipeline that were not considered in this study.

The comparative analysis between Linux and Windows agents revealed some interesting insights into the performance and energy efficiency of different operating systems. The results showed that Windows agents were more energy-efficient than Linux agents, with lower CPU and memory usage. This suggests that organizations may benefit from using Windows agents in their CI/CD pipelines to reduce energy consumption and improve resource utilization.

