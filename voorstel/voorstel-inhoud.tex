%---------- Inleiding ---------------------------------------------------------

\section{Introduction}%
\label{sec:introduction}
In the world of software development, the optimization of Continuous Integration and Deployment (CI/CD) processes has become a decisive concern for organizations striving to maintain efficiency and sustainability. This bachelor's thesis focuses on a concrete problem situation within a company context, with the center of attention on a case study.

The target audience for this research encompasses IT professionals, especially those who are engaged in CI/CD processes. By focusing this research on this specific group, it aims to provide insights and solutions to system administrators who already have knowledge of the CI/CD world.

The case study guiding this research is Wolters Kluwer Financial Services FRR. It's an organization grappling with challenges related to resource optimization, efficiency, and environmental impact within their CI/CD infrastructure, TeamCity. The company is looking for more sustainable and resource-friendly solutions due to bad configurations in the past.

The central research question steering this thesis is: "How can CI/CD processes be optimized to make them more environmentally sustainable while maintaining efficiency
and resource optimization?"

Through a thorough analysis and proof-of-concept, the aim is to contribute practical solutions to the identified problem. The goal is not only to enhance CI/CD efficiency but also to align this process with principles of environmental sustainability.

The success of this research will be gauged by the impact our recommendations have on Wolters Kluwer's CI/CD processes. Whether indicated through a proof-of-concept implementation, a comprehensive report featuring actionable recommendations, or other practical outcomes, the benchmark for success lies in positively transforming CI/CD practices toward sustainability.
%---------- Stand van zaken ---------------------------------------------------

\section{State-of-the-art}%
\label{sec:state-of-the-art}

In examining the present landscape of software automation, the breadth of available options mirrors the vastness of an ocean. \autocite{King2019} The ever-evolving landscape of automation is complex, and if you want to utilize all possible solutions, you may be busy for a long while. \autocite{King2019} This literature review aims to provide an overview of the current state of CI/CD processes, drawing extensively from professional literature.

\subsection{CI/CD Infrastructure and Efficiency}

It all initiates from the outset: which tool shall be utilized? It is important to consider multiple factors, like features relevant to your needs, the ease of use, scalability and cost of the tool. Furthermore, it is important to break the pipeline down into multiple phases, this can dismantle bottlenecks and accelerates the delivery process. The article of \textcite{Dahunsi2023} further delves into the significance of automated testing and Quality Assurance, paralleling builds for scalability, implementing monitoring and feedback loops, conducting security checks and embracing a culture of continuous improvement using the Deming Cycle. The journey to an optimal CI/CD pipeline is a perpetual quest in the ever-evolving landscape of DevOps.

\subsection{Sustainability in CI/CD}

In the wake of escalating environmental concerns, \textcite{Corewave2023}'s newsletter underscores the imperative for sustainable practices in software and app development. Highlighting eco-friendly strategies, the piece emphasizes code optimization, cloud computing efficiency, energy-efficient algorithms, and user education as key contributors to a more sustainable approach. Beyond environmental benefits, embracing sustainable development offers businesses cost savings, economic advantages, improved user experiences, and an enhanced brand reputation. Case studies featuring tech giants like Google, Apple, and Ecosia illustrate successful implementations of sustainable initiatives, reinforcing the notion that integrating eco-friendly methods into software and application development is not just a trend but a crucial step towards a more resilient and sustainable future.

\subsection{Shifting to Linux}

The shift from Windows agents to Linux Virtual machines not only offers better cost-effectiveness and operational efficiency but also underscores a significant contribution to environmental sustainability. \autocite{Germain2017} Linux's open-source nature, coupled with its lightweight yet powerful design, promotes efficient use of resources and energy. Code optimization practices in Linux development play a significant role in reducing energy consumption, thereby promoting environmental consciousness. Linux's compatibility with cloud computing also contributes to energy efficiency by enabling the selection of providers that prioritize eco-friendly data centers and practices like serverless computing. Moreover, Linux's emphasis on energy-efficient algorithms and user education reinforces its reputation as a platform with a reduced environmental footprint. Opting for Linux Virtual Machines over Windows agents not only benefits businesses in terms of operational costs but, also actively supports sustainability initiatives, thereby positively impacting the environment.

Ethan T. Schmidt, chief technology officer at GymBull.com, shifted to Linux and never looked back. Although there was some initial apprehension from his non-developer staff about leaving Windows behind, the transition was almost transparent. The Ubuntu interface of today is virtually the same as Windows, and it helps that they have Linux troubleshooters to help out anyone who needs it. \autocite{Germain2017}

% Voor literatuurverwijzingen zijn er twee belangrijke commando's:
% \autocite{KEY} => (Auteur, jaartal) Gebruik dit als de naam van de auteur
%   geen onderdeel is van de zin.
% \textcite{KEY} => Auteur (jaartal)  Gebruik dit als de auteursnaam wel een
%   functie heeft in de zin (bv. ``Uit onderzoek door Doll & Hill (1954) bleek
%   ...'')


%---------- Methodologie ------------------------------------------------------
\section{Methodology}%
\label{sec:methodologie}

\subsection{Phase 1: Literary review}
The initial phase of my research involves conducting a comprehensive review of academic literature and web articles pertinent to sustainability, CI/CD processes, and their eco-friendly practices. This will be achieved through targeted searches using relevant terms and keywords. Upon completion of this study, my aim is to gain deeper insights into the evolution of CI/CD processes and their alignment with sustainability principles. Specifically, I intend to acquire a better understanding of the various strategies, tools, and approaches utilized for integrating sustainability into CI/CD pipelines.

\subsection{Phase 2: Shortlist}
Following the literature review, the next step involves validating the identified information through practical observation at Wolters Kluwer. This entails examining their CI/CD processes and conducting interviews with colleagues to gather qualitative insights into sustainable CI/CD practices. By leveraging this firsthand information, I aim to compile a shortlist of the most promising sustainability practices, frameworks, and tools based on their practical applicability. In making this shortlist, factors such as scalability and ease of adoption, as mentioned in the state-of-the-art literature, will be carefully considered.

\subsection{Phase 3: Metrics and Tools Analysis}
A set of sustainability metrics and key performance indicators relevant to CI/CD processes will be developed from the evaluations conducted. Existing tools and software designed to support more sustainable CI/CD processes will be evaluated based on these metrics, enabling the identification of solutions that align closely with the defined sustainability criteria. This evaluation process will facilitate informed decision-making regarding the adoption and implementation of sustainable CI/CD practices.

\subsection{Phase 4: Proof-of-Concept}
The subsequent phase of the bachelor thesis involves crafting a comprehensive plan and a Proof-of-Concept that delineates the objectives, scope, and metrics to be assessed. Criteria for success will be defined to evaluate the effectiveness of each shortlisted approach. Subsequently, the Proof-of-Concept will be implemented within Wolters Kluwer's controlled CI/CD environment. This implementation phase will enable the measurement and collection of data pertaining to environmental impact, efficiency, and other relevant metrics. Concurrently, this initiative will aid Wolters Kluwer in advancing towards more sustainable and efficient CI/CD practices.

\subsection{Phase 5: Conclusion}
The findings from the comparative study of each shortlisted approach will be summarized to draw conclusions regarding the most effective and practical, sustainable CI/CD practices. This summarization will enable the provision of recommendations for organizations, such as my internship company, that are seeking to implement sustainable CI/CD processes. These recommendations will be based on the identified strengths and weaknesses of each approach, facilitating informed decision-making and effective implementation strategies for sustainable CI/CD practices.

\begin{figure}
    \centering
    \includegraphics[width=0.8\linewidth]
    {graphics/Gantt.png}
    \caption{\label{fig:protocol}Gannt-diagram of the methodology}
\end{figure}

%---------- Verwachte resultaten ----------------------------------------------
\section{Expected results, conclusion}%
\label{sec:verwachte_resultaten}

The conclusion of the comparative study is expected to offer several key insights. Firstly, the identification of optimal sustainability practices for CI/CD processes sheds light on why certain approaches outperform others in reducing environmental impact and resource consumption. The findings will also address the scalability of each shortlisted sustainability approach, providing insights into their suitability for different organizational scales.

Additionally, the study will contribute recommendations for organizations aiming to implement sustainable CI/CD processes. These recommendations will encompass best practices, potential challenges, and considerations for successful adoption. The delicate balance between CI/CD efficiency and sustainability goals will be acknowledged, offering guidance on optimizing both aspects. Ultimately, the study's contributions aim to advance the field of sustainable software development, providing valuable insights for practitioners, researchers, and organizations seeking to align their CI/CD processes with environmental sustainability.









